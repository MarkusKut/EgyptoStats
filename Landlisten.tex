% Options for packages loaded elsewhere
\PassOptionsToPackage{unicode}{hyperref}
\PassOptionsToPackage{hyphens}{url}
%
\documentclass[
]{article}
\usepackage{amsmath,amssymb}
\usepackage{iftex}
\ifPDFTeX
  \usepackage[T1]{fontenc}
  \usepackage[utf8]{inputenc}
  \usepackage{textcomp} % provide euro and other symbols
\else % if luatex or xetex
  \usepackage{unicode-math} % this also loads fontspec
  \defaultfontfeatures{Scale=MatchLowercase}
  \defaultfontfeatures[\rmfamily]{Ligatures=TeX,Scale=1}
\fi
\usepackage{lmodern}
\ifPDFTeX\else
  % xetex/luatex font selection
\fi
% Use upquote if available, for straight quotes in verbatim environments
\IfFileExists{upquote.sty}{\usepackage{upquote}}{}
\IfFileExists{microtype.sty}{% use microtype if available
  \usepackage[]{microtype}
  \UseMicrotypeSet[protrusion]{basicmath} % disable protrusion for tt fonts
}{}
\makeatletter
\@ifundefined{KOMAClassName}{% if non-KOMA class
  \IfFileExists{parskip.sty}{%
    \usepackage{parskip}
  }{% else
    \setlength{\parindent}{0pt}
    \setlength{\parskip}{6pt plus 2pt minus 1pt}}
}{% if KOMA class
  \KOMAoptions{parskip=half}}
\makeatother
\usepackage{xcolor}
\usepackage[margin=1in]{geometry}
\usepackage{graphicx}
\makeatletter
\def\maxwidth{\ifdim\Gin@nat@width>\linewidth\linewidth\else\Gin@nat@width\fi}
\def\maxheight{\ifdim\Gin@nat@height>\textheight\textheight\else\Gin@nat@height\fi}
\makeatother
% Scale images if necessary, so that they will not overflow the page
% margins by default, and it is still possible to overwrite the defaults
% using explicit options in \includegraphics[width, height, ...]{}
\setkeys{Gin}{width=\maxwidth,height=\maxheight,keepaspectratio}
% Set default figure placement to htbp
\makeatletter
\def\fps@figure{htbp}
\makeatother
\setlength{\emergencystretch}{3em} % prevent overfull lines
\providecommand{\tightlist}{%
  \setlength{\itemsep}{0pt}\setlength{\parskip}{0pt}}
\setcounter{secnumdepth}{-\maxdimen} % remove section numbering
\ifLuaTeX
  \usepackage{selnolig}  % disable illegal ligatures
\fi
\usepackage{bookmark}
\IfFileExists{xurl.sty}{\usepackage{xurl}}{} % add URL line breaks if available
\urlstyle{same}
\hypersetup{
  pdftitle={Landlisten},
  hidelinks,
  pdfcreator={LaTeX via pandoc}}

\title{Landlisten}
\author{}
\date{\vspace{-2.5em}}

\begin{document}
\maketitle

Nach:

https://www.hermopolis.gwi.uni-muenchen.de/die-landlisten-von-hermopolis/

\section{Beschreibung}\label{beschreibung}

Die Landlisten informieren über Besitzer von Grundbesitz im
hermopolitischen Nomos, die im „Phrouriou Libos`` in Hermopolis selbst
und im benachbarten Antinoopolis wohnten. Die Listen bestehen aus den
Papyri P.Herm. Landl. 1 (= P.Giss. I 117 = P.Giss.inv. 4 A--H = G) und
P.Herm. Landl. 2 (= P.Flor. I 71 = F), die in die Zeit nach ca. 346/7
n.~Chr. datiert werden können. Die Eintragungen sind nach
Anfangsbuchstaben der Personen in alphabetischer Reihenfolge, separat
für die beiden Wohnorte, eingetragen. Außerdem werden der Pagus
(durchnummerierte Bezirke), der Status als Privat- oder Staatsland und
die Größe in Aruren (1 Arure = 2756 m²) angegeben. Zur Unterscheidung
der Personen wird der Vatername und/oder Beruf erwähnt.In den Papyri ist
in allen Pagi, außer dem 7., Besitz eingetragen, da dort vermutlich
Hermopolis selbst lag und es möglicherweise eine eigene Landliste für
die Stadt gab. Durch diese Papyri lassen sich zum einen Rückschlüsse auf
die Gesellschaftsstruktur in Hermopolis im 4. Jh. n.~Chr. ziehen und zum
anderen Genealogien aufstellen, wie es für Aurelia Charite und die
Familie des Hyperechios gilt, die auch durch andere Quellen bekannt
sind. In den beiden Städten gab es viele Beamte und Militärs. Es lässt
sich feststellen, dass der größte Teil der Fläche Privatland war und die
Anzahl der Landbesitzer sehr hoch war. Das Land war äußerst ungleich
verteilt: Nur vier Familien besaßen etwa 36\% des gesamten Landes,
während die Stadtbewohner zusammen ungefähr 25--30\% des Landes in
Hermopolites hielten. Außerdem war der Großteil des Landes um die Stadt
Hermopolis konzentriert.

\subsection{Literatur}\label{literatur}

R. S. Bagnall, Landholding in late Roman Egypt: the distribution of
wealth, in: JRS 82, 1992, 128-149.

A. K. Bowman, Landholding in the Hermopolite nome in the fourth century
A.D, in: JRS 75, 1985, 137-163.

M. Lewuillon-Blume, Enquête sur les registres fonciers (P. Landlisten):
essai sur les titres et professions, in: CdE 60 Fasc. 119-120, 1985,
138-146.

M. Lewuillon-Blume, Enquête sur les registres fonciers (PLandl.): la
repartition de la propriété et les familles de propriétaires, in: B. G.
Mandilaras (Hrsg.), Proceedings of the XVIII International Congress of
Papyrology: Athens 25-31 May 1986. Volume II, Athen 1988, 279-286.

P. J. Sijpesteijn, K. A. Worp, Zwei Landlisten aus dem Hermupolites (P.
Landlisten) (Studia Amstelodamensia ad epigraphicam, ius antiquum et
papyrologicam pertinentia 7), Zutphen 1978.

\end{document}
